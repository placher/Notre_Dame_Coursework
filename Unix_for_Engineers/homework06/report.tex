\documentclass{article}

\usepackage{graphicx}
\usepackage{hyperref}
\usepackage{float}
\restylefloat{table}

\title{Diversity in the Notre Dame Computer Science Department}
\author {J. Patrick Lacher \\ \href{mailto:jlacher1@nd.edu}{jlacher1@nd.edu}}

\begin{document}

	\pagenumbering{arabic}
	\maketitle
	
	\section{Overview}
	
	\paragraph{}
	For this assignment, I developed a script \textbf{demo.sh} that downloads demographic data from the professor's computer and generates \textbf{.dat} files with the formatted data. This data is used by two \textbf{.plt} scripts to generate the gnuplots seen below. Finally, this report was generated by compiling the file \textbf{report.tex}. All of these actions, as well as a cleanup utility, are automated by a makefile.
	\paragraph{}
	Looking at the assembled data, it is clear that the CS department at Notre Dame has a similar demographic to that of the overall industry: predominantly white males. There are roughly twice as many male students as female students in our graduating year, and this represents a dramatic closing of the gap, compared to the roughly 5:1 ratio in 2013. Ethnicity is another story, with the CS population being overwhelmingly Caucasian, with no real trend to show any kind of change.
	
	\section{Methodology}
	
	\paragraph{}
	The demographics.csv file is downloaded and processed by the \textbf{demo.sh} script. The script uses two \textbf{awk} commands to seperate the gender and ethnic demographic data into two respective \textbf{.dat} files for use in generating GNUplot images. Since all of the data was organized neatly in columns, it was simple to have the script count occurances of markers within various columns (\textbf{M}/\textbf{F}, etc.) and return the results in a neatly formatted fashion. A few \textbf{sed} commands were also inserted into the two pipelines to sanitize the input/output.

	\section{Analysis}
	
	\paragraph{}
	Table 1 and Figure 1 display gender distribution information for the last 6 graduation years, hilighting the fact that men make up more than double the percentage of the student body than women.
	
	\begin{table}[H]
	
		\centering
		
		\caption{Gender Distribution of ND Computer Science Students}
		
		\label{tab:table1}
		
		\begin{tabular}{l||c|c}
				
			Date & Female & Male\\
				
			\hline
				
			2013 & 14 & 49\\
			
			2014 & 12 & 44\\
			
			2015 & 16 & 58\\
			
			2016 & 19 & 60\\
			
			2017 & 26 & 65\\
			
			2018 & 36 & 90\\
				
		\end{tabular}
			
	\end{table}
	
	\begin{figure}[H]
		\includegraphics[width=\linewidth]{gender.png}
		
		\caption{Gender Distribution of ND Computer Science Students}
		
		\label{fig:figure1}
		
	\end{figure}
	
	\paragraph{}
	Table 2 and Figure 2 display ethnic distribution information for the last 6 graduation years, showing that an overwhelming majority of Computer Science students are Caucasian and that the relative distribution has been nearly constant over the data period.
		
	\begin{table}[H]
	
		\centering
		
		\caption{Ethnic Distribution of ND Computer Science Students}
		
		\label{tab:table2}
		
		\begin{tabular}{l||c c c c c c}
				
			 & 2013 & 2014 & 2015 & 2016 & 2017 & 2018\\
			
			\hline
				
			Native American & 1 & 1 & 1 & 7 & 5 & 4\\
            
            African American & 3 & 2 & 4 & 1 & 5 & 3\\
            
            Oriental & 7 & 5 & 9 & 9 & 12 & 8\\
            
            Caucasian & 43 & 43 & 47 & 53 & 60 & 91\\
            
            Hispanic & 7 & 4 & 10 & 9 & 3 & 12\\
            
            Multiple & 2 & 1 & 1 & 0 & 6 & 8\\
            
            Undeclared & 0 & 0 & 2 & 0 & 0 & 0\\ 
				
		\end{tabular}
			
	\end{table}
				
	\begin{figure}[H]
		\includegraphics[width=\linewidth]{ethnicity.png}
		
		\caption{Ethnic Distribution of ND Computer Science Students}
		
		\label{fig:figure2}
		
	\end{figure}		
	
	\section{Discussion}
	
    \paragraph{}
    Diversity is essential to a strong educational enviroment. With diversity, we find greater exposure to ideas and experiences vastly different from our own, better enabling us to form a complex understanding of the world and how to approach the problems it will eventually force us to face. The department, and the technology industry as a whole, should work to foster a culture of diversity that will help us all to find innovative and complex solutions to the complicated problems that we face and will face in the world.
    \paragraph{}
    Being one of the white males that dominate the field and our graduating class, I may not have the best perspective to answer questions about the current openness and suppotiveness of the Computer Science department. In my experience over the last two years, though, I have found the department to be exceedingly supportive of its students. The professors are some of the most easily accessable of any department on campus and are highly sympathetic to the needs of and issues faced by their students. I've had a great experience as a Notre Dame CS student, and my advice to the department would be to continue the good work they're already doing.
    \paragraph{}
    I've experienced no real quantifiable challenges in my time in the Computer Science department. The only thing I can think of is the occasional problem with fully grasping some new concept or piece of new material, and the department has been more than adequately supportive in the widespread availability of professor's office hours and the overall helpfulness of TA's and grad students. Notre Dame has done a great job of fostering a supportive CS community that I am happy to be a part of.

\end{document}