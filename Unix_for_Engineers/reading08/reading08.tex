\documentclass{article}

\usepackage{graphicx}

\title{Reading 08: Document Tools}
\author {J. Patrick Lacher}
\date{}

\begin{document}

	\pagenumbering{arabic}
	\maketitle
	
	\section{Overview}
	
	For this experiment I created three scripts:
	
	\begin{itemize}
		\item roll dice.sh: This script simulates rolling a dice.
		\item experiment.sh: This script uses roll dice.sh to perform an experiment and then collect that data
into results.dat.
		\item histogram.plt: This script uses gnuplot to create a graph of the data in results.dat.
	\end{itemize}
	
	\section{Rolling Dice}
	
	First, I created a script called roll dice.sh that uses the shuf command to simulate rolling a die with a certain number of sides for a specified amount of rolls.

	\$ ./roll\_dice.sh -h
	
	usage: roll\_dice.sh [ -r ROLLS -s SIDES ]
	
	
	-r ROLLS Number of rolls of die (default: 10)
	
	-s SIDES Number of sides on die (default: 6)

	\section{Experiment}
	
	Second, I created a script called experiment.sh that uses roll dice.sh to simulate rolling a six-sided die 1000 times. My script uses awk to collect the results into a single file called results.txt.
	
	\section{Results}
	
	Table 1 contains the results of my experiment of rolling a dice 1000 times:
	
		\begin{table}[h!]
	
			\centering
		
			\caption{Dice Rolling Results}
		
			\label{tab:table1}
		
			\begin{tabular}{l||c}
				
				Side & Counts\\
				
				\hline
				
				1 & 183\\
				
				2 & 165\\
				
				3 & 171\\
				
				4 & 170\\
				
				5 & 151\\
				
				6 & 160\\
				
			\end{tabular}
			
		\end{table}
Figure 1 contains a plot of my experimental results as produced by histogram.plt:
	
	\begin{figure}[h!]
		\includegraphics[width=\linewidth]{histogram.png}
		
		\caption{The results of 1000 dice rolls}
		
		\label{fig:results1}
		
	\end{figure}

\end{document}